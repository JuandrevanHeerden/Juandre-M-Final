\appendix
\chapter{ ESTABLISHING AN INFORMATION SECURITY CULTURE IN
ORGANISATIONS: AN OUTCOMES BASED EDUCATION APPROACH}

Johan van Niekerka and Rossouw von Solmsb

aDepartment of Business Information Systems, Port Elizabeth
Technikon bDepartment of Information Technology, Port Elizabeth
Technikon

ajohanvn@petech.ac.za, 041 5043048, Private Bag X6011, PORT
ELIZABETH, 6000 brossouw@petech.ac.za, 041 5043604, Private Bag
X6011, PORT ELIZABETH, 6000

\section{ABSTRACT} In today's business world information is a
valuable commodity and such needs to be protected. This protection
is typically implemented in the form of various security controls.
In order for these controls to be effective the users in the
organization needs to be educated about these controls. Many
recent studies have shown that user education alone is not
sufficient and that the emphasis should be on the creation of an
organizational culture of information security. This paper
examines the concept of organizational cultures and the role
education plays in the establishment of such a culture. It then
examines outcomes based education (OBE) and finally attempts to
show how OBE can play a positive role in organizations wanting to
establish a culture of information security.

KEY WORDS Information Security, Information Security Culture,
Outcomes Based Education, Awareness


ESTABLISHING AN INFORMATION SECURITY CULTURE IN ORGANISATIONS: AN
OUTCOMES BASED EDUCATION APPROACH

1.  INTRODUCTION

In today's business world information is a valuable commodity and
such needs to be protected. It affects all aspects of today's
businesses from top management right down to operational level. In
order to avoid loss or damage to this valuable resource, companies
need to be serious about protecting their information. This
protection is typically implemented in the form of various
security controls. However, it is very difficult to know exactly
which controls would be required in order to guarantee a certain
acceptable minimum level of security. Furthermore, managing these
controls to see that they are always up to date and implemented
uniformly throughout the organization is a constant headache to
organizations.

There exist several internationally accepted standards and codes
of practice to assist organizations in the implementation and
management of an organizational information security strategy.
Some of the better known examples would include the BS7799
(British Standards Institute, 1999) and the Guidelines for the
Management of Information Technology Security (GMITS) (GMITS,
1995).

These standards and codes of practice provide organizations with
guidelines specifying how the problem of managing information
security should be approached. One of the key controls identified
by all the major IT Security standards published to date is the
introduction of a corporate information security awareness
program. The purpose of such a program is to educate the users
about Information Security or, more specifically, to educate users
about the individual roles they play in the effectiveness of one
type of control, namely, operational controls.

2.  THE HUMAN SIDE OF INFORMATION SECURITY

Information Security controls can generally be sub-divided into
three categories: Physical controls, Technical controls and
Operational controls. Physical controls deal with the physical
aspects of security, for example; a physical control might state
that an office containing sensitive documents should have a lock
on the door. Technical controls are controls of a technical
nature, for example; forcing a user to authenticate with a unique
username and password before allowing the user to access the
operating system would be a technical control. The third category,
operational controls, consists of all controls that deal with
human behaviour.

Employees, whether intentionally or through lack of knowledge, are
the greatest threat to Information Security (Thomson, 1998) and
because these operational controls rely on human behaviour, hence
employee behaviour, they can be seen as the weakest link in
Information Security. Unfortunately both physical and technical
controls rely heavily on these operational controls for
effectiveness. As an example, an operational control might state
that a user leaving their office must logoff from the operating
system and lock their office door. If a user were to ignore this
control, both the technical control forcing authentication and the
physical control of having a lock on the door would be rendered
useless. Thus anyone who thinks that security products, i.e.
technical and physical controls, alone offer true security is
settling for the illusion of security (Mitnick \& Simon, 2002).

According to Dhillon (1999), the user education program is singled
out because increasing awareness of security issues is the most
cost-effective control that an organization can implement. This
control is so cost-effective because it ensures that all users are
aware of the operational controls without which all other controls
cannot operate efficiently. Special care should be taken that the
awareness program is presented in such a form that it do not go
beyond the comprehension of the average user. The emphasis should
be to build an organizational sub-culture of security awareness.

Many recent studies have shown that the establishment of an
Information Security "culture" in the organization is in fact
necessary for effective Information Security (Eloff \& Von Solms,
2000; Von Solms, 2000). Some of these studies have presented
definitions of what an Information Security culture is, but
currently there exists very little knowledge on how such a culture
can be established. It is however clear that the user education
program will have to play a major role in the establishment of
such a culture.

3.  ESTABLISHING A CULTURE OF INFORMATION SECURITY

Before the task of establishing an Information Security culture
can be tackled it is necessary to understand what such a culture
is. According to The American Heritage Dictionary of the English
Language (2000) a culture is: "   The totality of socially
transmitted behavior patterns, arts, beliefs, institutions, and
all other products of human work and thought "   The predominating
attitudes and behavior that characterize the functioning of a
group or organization.

In terms of Information Security a corporate culture of
Information security can thus be seen as the predominating
attitudes towards information security and security related
behaviour that characterize the functioning of the employees
within the organization. Thus, in an organization that has a
culture of information security, the employees would adhere to
proper security practices during execution of their day-to-day
functions because that is simply the way things are done. It is
obvious that in order for employees to be able to adhere to proper
security practices the employees would have to know what proper
security practices are. Therefore information security education
would have to play a key role in the establishment of such a
culture.

Even though user education is essential for the establishment of a
successful corporate culture of information security, education on
its own cannot change a corporate culture. Sadri and Lees (2001)
place the main responsibility for changing a corporate culture on
the shoulders of management.

Corporate culture should start with proper visionary statements
and should thereafter be positively reinforced through management
behaviour i.e. rewarding employees' successes and distributing
newsletters and videos that reinforce the culture. Leadership from
the very top of an organization is essential for major cultural
change. However, even though middle managers do not initialize the
cultural change, ultimately it is their actions that produce the
changes (Brubakk \& Wilkinson, 1996).

Thus, in order to establish a culture of information security, the
top management of the organization will have to start the process
via vision and mission statements, as well as policy changes. This
process has to be supported by a sound user education program and
reinforced via continuous feedback. This feedback will come from
the organization's middle management (Brubakk \& Wilkinson, 1996).

4.  OUTCOMES BASED EDUCATION

Since the aim of the user education program is not to prepare the
users for further levels of formal education, but rather to help
them achieve information security know-how for use in their
everyday jobs, the educational methodology used should be chosen
accordingly. Outcomes Based Education (OBE) might in fact be
ideally suited for use in such programs since the aim of OBE is to
help learners achieve a specific outcome, in this case information
security awareness.

OBE is defined as an approach to teaching and learning which
stresses the need to be clear about what learners are expected to
achieve. The educator states beforehand what "outcome" is expected
of the learners. The role of the educator is then to help the
learners achieve that outcome (Siebörger, 1998).

Outcomes can be defined as either cross-curriculum (general
outcomes) or specific outcomes. A cross-curriculum outcome can be
seen as the desired effect that attaining a specific competency
should have within the general environment within which the
learner operates. A specific outcome is one that directly
demonstrates the mastery of the appropriate skill that the learner
should gain from the OBE program.

For each outcome an assessment standard should be defined. These
standards are necessary in order to provide feedback to the
learners. According to Siebörger (1998) assessment is essential to
OBE to measure the degree to which a learner has achieved an
outcome. In fact being able to assess progress and provide
feedback to the learner is a prerequisite for any educational
program to be successful. Fingar (1996) states that feedback,
specifically in the form of knowledge regarding the outcomes of
the learners' actions, is required for learning to take place.
Furthermore this feedback should be continuous and constructive
(DOE, 2001).

The educational process in general can be viewed as a system of
teaching and learning activities that are tied together via
various feedback loops. It also includes other functions such as
assessment, admission, quality assurance, direction and support
(Tait, 1997). All of these components can, and should, play a role
in the creation of an effective Information Security education
program.

5.  OUTCOMES BASED EDUCTION FOR INFORMATION SECURITY

Up to this point this paper has shown that establishing a
corporate culture will have to start with top management, be
continuously reinforced via feedback from middle management to the
employees and should be based on a sound user education program.
It has also been suggested that this user education program should
be outcomes based.

The following example will attempt to show how these components
could be brought together in order to affect such a cultural
change. Suppose the controls for which we want to change the
users' behaviour deals with proper password usage.

Firstly, top management will have to show their commitment to
information security by developing visionary statements and/or
slogans (Sadri and Lees, 2001). This could be part of the
corporate vision statement or simply some posters stating that the
organization is committed to improving information security. For
example, a poster, signed by the CEO, could be posted throughout
the organization stating: "At ABC we are committed to the
integrity, availability and confidentiality of all our
information". These visionary statements have to be followed up by
a corporate information security policy. One of the policy
statements would be: "All users of information must be
authenticated before being allowed to use information resources".
The policy in turn should be supported by a set of procedures
dealing with the specific operational control. Thus for password
usage these procedures could include: "   All users must use
passwords that is at least eight characters long and include at
least two non-alphabetic characters "   All users must change
their passwords at least once every two weeks "   A user may not
write down their password or share their password with any other
user

Secondly, a user education program should be constructed to
educate the users about proper password usage. The specific
outcomes for such a program could include: "   The learner (user)
should know why using a properly constructed password is necessary
"   The user should be able to demonstrate constructing their own
passwords and recalling them with the aid of simple mnemonic
techniques "   The user should be able to change their own
passwords on the required system "   The user should know that
they are accountable for any misdeeds that takes place using their
authentication information and should be fully aware of the
possible consequences to themselves and the organization. General
(cross-curriculum) outcomes could include: "   All users'
passwords should be changed once per week "   All users' passwords
should withstand a standard brute force dictionary attack.

Thirdly, middle management will have to positively reinforce any
learning that took place by giving continuous feedback to the
users. For the specific outcomes feedback can form part of the
actual education program, possibly in the form of a quiz or small
workshop at the end of a learning session. Feedback for the
general outcomes however, should be seen as drivers for the
desired cultural change. It should be fairly easy to gather
metrics such as the percentage of users in the organization that
changed their passwords for a specific week and the percentage of
passwords cracked by an in-house test attack. These metrics could
be gathered per department, branch, etc. and can then easily be
made part of the key performance indicators for the appropriate
middle level manager. The old adage that what you measure is what
you get will then play its part by ensuring that the appropriate
line managers will feed this statistics back to their staff since
it impacts on their own performance evaluations.

The process discussed in this section would have to be repeated
for each control addressed in the corporate information security
policy or sub-policies. Obviously separate visionary statements
and/or slogans would not be required for each control. However,
each control should be addressed specifically in an information
security policy, and by supporting procedures. Each control would
have to be included in an educational program. The outcomes and
measurables for these outcomes would have to be clearly defined.
Special care should be taken when defining the cross-curriculum or
general outcomes, since in this approach these general outcomes
act as drivers for the intended cultural change.

6.  CONCLUSION

This paper discussed aspects that will contribute to the
establishment of a corporate sub-culture of information security.
It showed that establishing a culture will have to start with top
management and should be continuously reinforced via feedback from
middle management to the employees. It also suggested that a sound
user education program should form this basis of such a cultural
change. The concept of outcomes based education (OBE) was examined
and then proposed as a very suitable methodology for use in the
establishment of such a culture.

This paper forms part of an ongoing research project at the Port
Elizabeth Technikon. In the paper a brief example of how OBE could
be used to assist organizations in the establishment of an
information security culture was given. This example is by no
means complete at this stage and only served to illustrate the
point that OBE can play a useful role in information security
education. It should be obvious that several factors still need to
be taken into account. For example: "   How should the educational
material be presented and will this differ for users in different
levels of the organization? I.e. end users vs. top management. "
What forms of user profiling would be needed to ensure the
learning material is appropriate for the intended audience?
Obviously not all controls available in a standard such as ISO/IEC
17799 should be taught to every user. These and other issues are
currently be examined and it is hoped that the research project
this paper forms part of will be able to present a complete and
holistic methodology for Information Security Education that will
directly contribute towards the establishment of an culture of
information security in organizations.

7.  REFERENCES

British Standards Institute (1999), BS 7799 Part 1: Code of
Practice for Information Security Management (CoP), BSI, UK.

British Standards Institute (1999), BS 7799 Part 2:Specification
for information security management systems, BSI, UK.

Brubakk, B., Wilkinson, A. (1996). Agents of change? Bank branch
managers and the management of corporate culture change.
International journal of Service Industry Management, 7 (2), pp.
21-43.

Department of Defence (1985), Department of Defence Trusted
Computer Security Evaluation Criteria (TCSEC), DoD, Washington DC.

Dhillon, G. (1999) Managing and controlling computer misuse,
Information Management \& Computer Security, 7 (4), pp. 171-175.

DOE. (2001) Draft Revised National Curriculum Statement:
Technology Learning Area. Department of Education. Available at:
http://education.pwv.gov.za/DoE\_Sites/Curriculum/New\_2005/draft\_revised\_national\_curriculu.htm

Eloff, M.M., Von Solms S.H. (2000). Information security
management: A hierarchical framework for various approaches.
Computers and Security, 19(3): pp. 243-256.

Fingar, P. (1996). The blueprint for business objects. New York,
New York : SIGS Books \& Multimedia

Guidelines to the Management of Information Technology Security
(GMITS). (1995). Part 1, ISO/IEC, JTC 1, SC27, WG 1.

Guidelines to the Management of Information Technology Security
(GMITS). (1997). Part 2, ISO/IEC, JTC 1, SC27, WG 1.

Haag, S., Cummings, M. \& Dawkins, J. (2000). Management
Information Systems for the Information Age (2nd ed.). United
States of America : Irwin/McGraw-Hill.

King, S. (1998). Are you ready for a BS 7799 audit ?. BSI, UK.

Martins, A., Eloff, J.H.P. (2002) Assessing Information Security
Culture. ISSA 2002, Muldersdrift, South Africa, 10-12 July 2002.

Mitnick, K.D., Simon, W.L. (2002) The art of deception:
Controlling the human element of security. United States of
America : Wiley Publishing, Inc.

Sadri, G., Lees, B. (2001) Developing corporate culture as a
competitive advantage. Journal of Management Development, 20 (10).
Pp. 853-859.

Siebörger, R. (1998) Transforming Assessment: A guide for South
African teachers. Cape Town, RSA : JUTA.

Tait, B. (1997). Object Orientation in educational software.
Innovations in Education and Training International, 34 (3). pp.
167-173.

The American Heritage Dictionary of the English Language, Fourth
Edition. (2000) Houghton Mifflin Company, USA.

Thomson, M. (1998). The development of an effective information
security awareness program for use in an organization. Unpublished
master's thesis. Port Elizabeth Technikon, Port Elizabeth, South
Africa.

Von Solms, B. (2000) Information Security - The Third Wave?
Computers \& Security, 19 (7), pp. 615-620.

Von Solms, R. (1998) Information Security Management (1): why
information security is so important, Information Management \&
Computer Security, 6 (4), pp. 174-177.

Von Solms, R. (1998) Information Security Management (2):
guidelines to the management of information technology security
(GMITS), Information Management \& Computer Security, 6 (5), pp.
221-223.

Von Solms, R. (1998) Information Security Management (3): the Code
of Practice for Information Security Management (BS 7799),
Information Management \& Computer Security, 6 (5), pp. 224-225.

Acknowledgement: We would like to thank the National Research
Foundation for their financial support for research conducted
towards this paper.
