\chapter{Research Methodology}
\label{chap: Chapter 4}

The aim of this chapter is to provide the reader with information regarding the overall research design, as well as the research methods used.

The purpose of the study is to develop a model through which related research papers can be recommended. To refine the scope of the research, and highlight the contribution to the field, the problem is to investigate using a natural language processing approach. The dataset used was academic papers from the Information Security South Africa conference.

The aim of this chapter is to define the research approach to provide a model by which that natural language processing along with other information retrieval techniques can be combined to provide recommendations. The chapter will begin by describing in which paradigm it vests itself. Furthermore, the difference between inductive and deductive reasoning is discussed. Deductive reasoning will mainly be used to create the model. This chapter then describes the research methods employed and how they fit together to develop, create, and test the model.

The following section will discuss the research design followed in the study.

\section{Research design}

This research study has adopted a positivist approach. The positivist paradigm is based on facts and observation \cite{wilson2014essentials} and reduces the impact that the researcher’s interpretation has on the study.

Positivist studies usually come from a deductive research approach, whereas an inductive research approach is manifested within the philosophy of phenomenology \cite{saunders2007research}. The inductive approach can also be known as inductive reasoning, which begins to look for patterns from observations and theories. Moreover, inductive reasoning follows the bottom-up approach, that includes crafting theories or general conclusions from specific observations \cite{saunders2007research}.

For example, when searching for emerging themes in data inductive reasoning is used \cite{Fereday2006}. In contrast, deductive reasoning follows the top-down approach, starting with the general and moving to the specific. It uses facts and rules to arrive at its conclusions \cite{Fereday2006}.

The scientific research domain makes use of this deductive approach. It enables various theories that are tested. Observational hypotheses can be drawn from them and ultimately be compared to data \cite{bechtel2013philosophy}. In the past, scientific research methodologies employed this approach when collecting and evaluating data \cite{MARSDEN2018A1}.

In the next section, the methods that were followed in this research study are discussed.

\section{Methods} \label{ssec:meth}

This research study was primarily experimental in nature. It did not focus on the researcher’s interpretation of the data, but rather on the findings which could be drawn directly from the data. This research study employed a set of methods, which make up the methodology. These methods include a literature review, experimentation, a prototype, and the creation of a model. The upcoming sub-sections provide more detail about how these methods fit into the research objectives and milestones.

\subsection{Literature review} \label{ssec:prep}

A literature review is a process of going through various academic studies in search of information dealing with the topic at hand \cite{olivier2009information}. The literature review in this study focuses on various topics of recommender systems, document clustering, natural language processing and their application to recommend related papers using Information Security South Africa past conference papers. The literature review was conducted to meet secondary research objectives one and two.

\subsection{Experimentation} \label{ssec:exper}

As mentioned, the research took on the positivist philosophy. This study was experimental in nature. As \citeA{olivier2009information} discussed, an experiment can be conducted with the following goals in mind, namely to test a theory, to prove a theory, and to see whether something interesting happens, which is also known as an exploratory experiment.

Experiments have three main goals; they are (1) to explore a theory, (2) test a theory, and (3) prove a theory. It was later added by \citeA{olivier2009information} that it is common for experiments to compare two cases, an older solution to a newer one to see how they compare. In the context of this study, the algorithms and techniques identified were used to construct a prototype to which was used to observe the possibility of using recommender systems, document clustering and natural language processing techniques together to satisfy the primary objective.

Testing can be used to see whether a certain theory holds up against specific cases. In testing a single theory, one should conduct a limited experiment. This may be done to ’feel’ whether the theory is correct, to refine the theory even more, or to justify a full-scale experiment. For example, an academic paper recommender system can employ natural language processing techniques to achieve paper recommendations. If it is determined that the experiment holds true, the theory can either be further refined or a full-scale experiment can then begin. In contrast to testing a theory, proving a theory would be conducted to ensure without a doubt that a theory holds true. Proving the previous theory requires all outside factors, which can be a hindrance, to be removed.

Lastly, to see if something interesting happens. This experimental goal has very little structure and provides freedom to play around with certain ideas. For example, an academic paper recommender system can employ natural language processing techniques along with topic modelling algorithms to achieve paper recommendations. Such experiments do not have certain outcomes and are conducted with no given theory. As mentioned in the previous paragraph, the experiments had little structure and were accompanied by a prototype to maximise what there is to learn. The creation of the prototype and experiments were to satisfy the third research sub-objective.

\subsection{Prototyping} \label{ssec:prot}

In information technology the term prototype refers to a simplified program or system that serves as an example or demo of the full-scale program or system \cite{olivier2009information}. A prototype usually only has a few characteristics of the bigger system. The simplicity of the prototype is deliberate because only the study subject matter will be tested or demonstrated. In the research environment, the prototype research method cannot be used solely to constitute the research. In other words, it cannot be used as the icing on the cake; rather, it needs another method to lean on.

Working well with other methods, a prototype can be used in multiple roles. Commonly, there are four roles that a prototype can take: proof of concept, prototype for experimentation, prototype for conceptual clarity, and exploratory research. First, after proposing and constructing a new model or new concept, researchers build a prototype to prove the concept. In other words, the statement can be made that the concept can be implemented and works well in practice. The second role, prototype for experimentation, can be used to gather all the information about the prototype.

In general, information gathered can range from measuring the speed of the system and the quality of a model. However, measuring the speed or quality of a model alone does not make a big research contribution. The third role, prototype for conceptual clarity, is used when a certain concept or work is difficult to visualise. Developing the prototype with this role in mind forces the researcher to focus on the concepts at hand and helps in not overlooking certain details. Furthermore, after the construction of the prototype and its merit can be shown, it can be used as a proof of concept. Lastly, the final role that a prototype can play is in exploratory research.

In all of the other three roles the prototype is constructed to aid the research process. However, when a model, algorithm or concept is not new there are still lessons to be learnt by developing it. For example, in 2003 a new topic modelling algorithm was created called latent dirichlet allocation (LDA) and it achieved great success in the natural language processing (NLP) and information retrieval (IR) domains \cite{blei2003latent}. However, there were only a few papers available for the use of LDA in the recommender systems domain. This is a perfect example of incorporating LDA into a recommender system to see what lessons there are to learn. If a major issue is identified while constructing the prototype and it can be linked to the incorporation of LDA and recommender systems, this role of the prototype can be used to create new knowledge. The quality of the research will, however, be determined by the interesting data.

\subsection{Modeling} \label{ssec:model}

A model can be defined to capture the essence of the system or process \cite{olivier2009information}. In addition, a model needs to be expressed clearly and concisely. In the context of this study, the model was created as to achieve the primary research objective. The construction of the model commenced after the literature reviews, which satisfied sub-objectives one and two. It then became an iterative process between constructing the model, to build the prototype, and to do experiments on it. The feedback from the experimentation ensured amendments to the model.

Furthermore, \citeA{olivier2009information} states that a model captures the essence or core of a system or process. All of this while it ignores all the aspects that do not bring value. The model in this study was the main method, and the prototype was created to support it. An experiment can be used to validate a model. For example, defining a model that takes academic research papers that learns trends from the academic research papers. The model can be validated by creating a prototype. Where users can look for recommendations based on one of users papers, which the prototype has never seen. In theory, the model looks like it works; however, using an experiment, the model can be tested in practice \cite{Steenkamp2007}.

\subsection{Argumentation}

Argumentation can be seen as the thread which ties several statements together. As mentioned, arguments string from facts to create a premise on which conclusions can be based. An argument can be supported by other arguments, it can derail other arguments, or highlight main ideas \cite{walton2009argumentation}. 

This study used argumentation throughout the development of the model. Deductive argumentation was used to develop a theory to create and test a high-level model, diving deeper into which algorithms were needed to address the primary objective \cite{VanHaaften2016}.

\section{Summary}

This chapter has discussed two major points: (1) the paradigm in which the research finds itself, and (2) the collection of methods which the study employed to achieve the main research objective. The aim of the research was to explore the possibility of using natural language processing and information retrieval techniques to satisfy the research objective; thus making it an exploratory study.

The literature reviews reported on in Chapters \ref{chap: Chapter 2} and \ref{chap: Chapter 3} showed viable avenues and options to pursue. This finding ensured that the model would be based on pre-existing knowledge, establishing it in the domain of epistemology. The second half of the chapter covered the different methods used. A survey of the literature to identify trends and algorithms that could be used in the study was captured in Chapter \ref{chap: Chapter 3}. This was followed by creating a prototype of the proposed model and running various experiments with it, adjusting certain values every time.

In the next chapter, the construction of the conceptual model will be discussed.
